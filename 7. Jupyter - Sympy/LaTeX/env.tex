\begin{frame}[allowframebreaks]{¿Instalar? - virtualenv (par\'entesis)}\vspace{0pt}

Cuando queramos \textbf{compartir} software, debemos generar nuestro propio fichero. Obviamente, no vamos a escribir manualmente las librer\'ias que utilizamos. Vamos a hacer que Python lo haga por nosotros, para ello (desde la direcci\'on del software):

\vspace{5pt}

1. Instalamos virtualenv.

\begin{block}{Virtualenv$^{Primera \, vez}$}
	pip install virtualenv
\end{block}

2. Generamos nuestro propio ``ambiente virtual``.

\begin{block}{Generaci\'on del ambiente $^{Primera \, vez*}$}
	virtualenv env\\
	(si no funciona...) python -m venv env
\end{block}

\vspace{20pt}

3. Entramos al ambiente:

\begin{block}{Entrar}
	.\textbackslash env\textbackslash Scripts\textbackslash activate
\end{block}

4. Instalamos las librerías manualmente:

\begin{block}{Instalamos...}
	pip install <librer\'ias de nuestra aplicaci\'on>
\end{block}

5. Generamos el archivo.

\begin{block}{Archivo .txt}
	pip freeze > requirements.txt
\end{block}

6. Salimos del ambiente virtual.

\begin{block}{Salir}
	deactivate
\end{block}

\end{frame}