\begin{frame}[fragile]{Funciones}\vspace{10pt}

Una \textit{funci\'on} es un segmento de c\'odigo que desarrolla una tarea espec\'ifica. Las funciones se escriben com\'unmente al principio o al final del \textit{Sketch}, o \textit{Script}, y se \textbf{llaman} dentro de otras funciones.

\vspace{5pt}

Las ventajas de usar funciones son:

\begin{itemize}
	\item Ayudan a mantener la \textbf{organizaci\'on} del c\'odigo (conceptualiza el algoritmo escrito).
	\item Permiten probar ``tareas" \textbf{una sola vez}.
	\item Disminuye probabilidades de errores de programaci\'on.
	\item Permite reutilizar c\'odigo.
\end{itemize}

\begin{center}
\begin{lstlisting}
	TV nombre_funcion (TV var1, TV var2, ...) {
		...
		return resultados;	//Opcional...	
	}
\end{lstlisting}
\end{center}

\end{frame}