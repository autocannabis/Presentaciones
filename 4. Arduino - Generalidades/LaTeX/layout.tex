\begin{frame}[allowframebreaks]{Software - Layout}\vspace{0pt}

Arduino usa \textit{C++} como lenguaje de programaci\'on. A diferencia de Python, C++ es un lenguaje \textbf{est\'atico}: debes declarar la naturaleza de las variables y tener en cuenta el tama\~no que ocupar\'an \'estas en la memoria RAM.

\begin{table}[h!]
\begin{adjustbox}{max width=\textwidth}
\begin{tabular}{|l|r|l|}
\hline
\multicolumn{1}{|c|}{\textbf{Tipo variable (TV)}} & \multicolumn{1}{c|}{\textbf{Memoria usada $[bits]$}} & \textbf{Aplicaci\'on}                                                                           \\ \hline
void                                         & \multicolumn{1}{c|}{-}                      & Declaraci\'on de funciones                                                                      \\ \hline
byte                                         & 8                                           & N\'umero entero entre 0 y 255.                                                                  \\ \hline
int                                          & 16                                          & \begin{tabular}[c]{@{}l@{}}N\'umero entero entre -32767\\ y 32767.\end{tabular}                 \\ \hline
long                                         & 32                                          & \begin{tabular}[c]{@{}l@{}}N\'umero entero entre -2,147,483,648\\ y 2,147,483,648.\end{tabular} \\ \hline
float                                        & 32                                          & \begin{tabular}[c]{@{}l@{}}N\'umero real entre -3.4028325E+38\\ y 3.4028325E+38.\end{tabular}   \\ \hline
boolean                                      & 8                                           & Variable booleana: true/false.                                                                \\ \hline
char                                         & 8                                           & Caracteres ASCII.                                                                             \\ \hline
String                                       & \multicolumn{1}{c|}{-}                      & Cadena de texto.                                                                              \\ \hline
\end{tabular}
\end{adjustbox}
\end{table}

\vspace{20pt}

Un \textit{Sketch} com\'un de Arduino se compone de:

\begin{itemize}
	\item \textit{void} setup: acciones que se desarrollan \textbf{una} sola vez.
	\item \textit{void} loop: acciones que se repiten una y otra vez (similar al ciclo while).
\end{itemize}

\vspace{10pt}

Los comentarios dentro del algoritmo se hacen de la siguiente manera:

\begin{itemize}
	\item // Comentario: Comenta una sola l\'inea de c\'odigo.
	\item /* Comentario */: Comenta una o varias l\'ineas. 
\end{itemize}

\vspace{10pt}

IMPORTANTE: La sintaxis de \textit{C++} demanda que el final de cualquier l\'inea de c\'odigo sea con ``;".

\end{frame}