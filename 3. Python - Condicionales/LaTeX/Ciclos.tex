\begin{frame}[t]{Ciclos - for}\vspace{0pt}

El ciclo for es un ciclo \textbf{est\'atico}. Una vez le decimos cu\'antas vueltas deseamos que d\'e, no hay marcha atr\'as.

\begin{block}{Estructura}
	for i in <range(numero) o elemento>:
\end{block}

\vspace{5pt}

Existen dos formas de iteraci\'on: por n\'umero o por elemento (lista, tupla o diccionario). Cuando es por \underline{n\'umero}, el ciclo, por defecto, inicia desde 0 y termina en \textit{n\'umero - 1}. Sin embargo, podemos adaptarlo para que haga lo que nosotros queramos:

\begin{block}{Ciclo for - n\'umero}
	for i in range(inicio, 'fin', incremento)
\end{block}

Para \textbf{diccionarios}:

\begin{block}{Ciclo for - Diccionarios}
	for key, value in dic.items():
\end{block}

\end{frame}