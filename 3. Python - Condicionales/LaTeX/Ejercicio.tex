\begin{frame}[fragile]{Ejercicios}\vspace{10pt}

\begin{enumerate}

	\item Imprime una lista correspondiente a la mitad de un n\'umero \textit{n}, que pertence a los n\'umeros naturales, y que es dado por el usuario (INDIVIDUAL).
	\item Imprime un texto, proveniente de un bloc de notas, hasta que se encuentre con una letra espec\'ifica (INDIVIDUAL).
	\item Sup\'on que un sustrato tiene la siguiente composici\'on qu\'imica. Disminuye los elementos de manera aleatoria e incrementa, de manera proporcional, el restante (la idea es que la suma \underline{siempre} de 100) (GRUPAL).	

\end{enumerate}
\begin{center}
\begin{lstlisting}
	sustrato = {
		'Potasio': 20,
		'Nitrogeno': 40,
		'Fosforo': 20,
		'Restante': 20					
		}
\end{lstlisting}
\end{center}
\end{frame}